\centering \footnotesize \setlength{\tabcolsep}{4pt} \renewcommand{\arraystretch}{0.8}
\begingroup
\fontsize{7.0pt}{8.5pt}\selectfont
\begin{longtable}{lrrrrrrrr}
\caption*{
{\large Distribución de las variables númericas}
} \\ 
\toprule
  & Promedio & SD & P10 & Mediana & P90 & Razón P90/P10 & Asimetría & Correlación \\ 
\midrule\addlinespace[2.5pt]
{\bfseries Ingresos laborales por hora} & 8,578.87 & 13,902.31 & 2,333.33 & 4,855.29 & 17,012.92 & 7.29 & 8.86 & 1*** \\ 
{\bfseries Edad} & 39.08 & 13.11 & 23 & 37 & 58 &   & 0.49 & 0.081*** \\ 
{\bfseries N. menores de edad} & 0.84 & 0.97 & 0 & 1 & 2 &   & 1.16 & -0.062*** \\ 
{\bfseries N. personas mayores inactivas} & 0.13 & 0.39 & 0 & 0 & 1 &   & 3.33 & -0.006 \\ 
\bottomrule
\end{longtable}
\endgroup
\begin{minipage}{0.8\textwidth}
                                   \begin{tablenotes}[scriptsize,flushleft]
                                   \item \footnotesize \textit{Nota:} Esta tabla muestra un conjunto de estadísticas descriptivas de las variables continuas incluidas en el conjunto de datos.
                                   Las columnas de Promedio y SD muestran la desviación estándar de las variables.Mientras que las columnas P10, Mediana y P90 muestran los valores de la distribución en los percentiles 10, 50 y 90.
                                   La columna Razón P90/P10 muestra el cociente entre los percentiles 90 y 10.
                                   La columna Asimetría muestra el coeficiente de asimetría de la distribución.
                                   Finalmente la columna Correlación muestra el valor de el Coeficiente de Correlación de Pearson con la variable de Ingresos laborales por hora.
                                   El tamaño de la muestra es de 14,632 observaciones.
                                   \end{tablenotes}
                                   \end{minipage}
