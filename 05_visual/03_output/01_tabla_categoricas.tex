\centering \footnotesize \setlength{\tabcolsep}{4pt} \renewcommand{\arraystretch}{0.8}
\begingroup
\fontsize{7.0pt}{8.5pt}\selectfont
\begin{longtable}{crrrl}
\toprule
  & N & \% & Promedio & SD \\ 
\midrule\addlinespace[2.5pt]
\multicolumn{5}{l}{{\bfseries Sexo}} \\[2.5pt] 
\midrule\addlinespace[2.5pt]
Masculino & 7,697 & 52.6\% & 8,926 & 14,830 \\ 
Femenino & 6,935 & 47.4\% & 8,193 & 12,785 \\ 
\midrule\addlinespace[2.5pt]
\multicolumn{5}{l}{{\bfseries Máximo nivel educativo}} \\[2.5pt] 
\midrule\addlinespace[2.5pt]
Secundaria completa & 4,748 & 32.4\% &  5,144 &  4,621 \\ 
Terciaria & 6,076 & 41.5\% & 14,004 & 19,833 \\ 
Primaria completa & 1,342 &  9.2\% &  4,271 &  2,825 \\ 
Primaria incompleta &   661 &  4.5\% &  3,678 &  2,048 \\ 
Secundaria incompleta & 1,708 & 11.7\% &  4,411 &  2,925 \\ 
Ninguno &    97 &  0.7\% &  3,234 &  2,694 \\ 
\midrule\addlinespace[2.5pt]
\multicolumn{5}{l}{{\bfseries Formalidad}} \\[2.5pt] 
\midrule\addlinespace[2.5pt]
Formal & 8,864 & 60.6\% & 11,002 & 16,351 \\ 
Informal & 5,768 & 39.4\% &  4,855 &  7,523 \\ 
\midrule\addlinespace[2.5pt]
\multicolumn{5}{l}{{\bfseries Posición ocupacional}} \\[2.5pt] 
\midrule\addlinespace[2.5pt]
Obrero o empleado del gobierno &   569 &  3.9\% & 17,157 & 12,246 \\ 
Obrero o empleado de empresa particular & 8,656 & 59.2\% &  8,633 & 13,136 \\ 
Trabajador por cuenta propia & 4,367 & 29.8\% &  7,307 & 13,341 \\ 
Empleado doméstico &   559 &  3.8\% &  4,089 &  1,670 \\ 
Patrón o empleador &   472 &  3.2\% & 14,438 & 28,920 \\ 
Otro &     8 &  0.1\% &  3,334 &  2,961 \\ 
Jornalero o peón &     1 &  0.0\% &  4,375 &     NA \\ 
\midrule\addlinespace[2.5pt]
\multicolumn{5}{l}{{\bfseries Cantidad de trabajadores
de la empresa en que trabaja}} \\[2.5pt] 
\midrule\addlinespace[2.5pt]
Independiente & 3,527 & 24.1\% &  6,081 & 12,200 \\ 
2-5 trabajadores & 2,704 & 18.5\% &  5,809 & 10,224 \\ 
6-10 trabajadores & 1,013 &  6.9\% &  6,416 &  7,555 \\ 
11-50 trabajadores & 1,875 & 12.8\% &  8,654 & 14,184 \\ 
>50 trabajadores & 5,513 & 37.7\% & 11,907 & 16,370 \\ 
\midrule\addlinespace[2.5pt]
\multicolumn{5}{l}{{\bfseries Oficio}} \\[2.5pt] 
\midrule\addlinespace[2.5pt]
Administración y gestión & 2,789 & 19.1\% & 11,754 & 18,746 \\ 
Agricultura, pesca y oficios rurales &   159 &  1.1\% &  4,738 &  5,083 \\ 
Arte, deporte y medios &    52 &  0.4\% &  9,287 &  6,600 \\ 
Comercio y ventas & 2,269 & 15.5\% &  6,635 & 11,679 \\ 
Educación, religión y cultura &   912 &  6.2\% & 15,430 & 20,177 \\ 
Industria y construcción & 2,306 & 15.8\% &  4,929 &  3,638 \\ 
Operarios y trabajos no calificados & 1,184 &  8.1\% &  4,851 &  3,031 \\ 
Profesionales científicos y técnicos & 1,629 & 11.1\% & 18,054 & 21,398 \\ 
Servicios personales y de seguridad & 3,193 & 21.8\% &  4,742 &  3,960 \\ 
Textiles y manufactura artesanal &   139 &  0.9\% &  5,185 &  2,332 \\ 
\bottomrule
\end{longtable}
\endgroup

NA
NA
\begin{minipage}{0.8\textwidth}
                                   \begin{tablenotes}[scriptsize,flushleft]
                                   \item \footnotesize \textit{Nota:} Esta tabla muestra la cantidad de observaciones en cada nivel de cada variable, 
                                   así como la proproción que representan del total de distribución.
                                   Las columnas de Promedio y SD, muestran respectivamente, el promedio y la distribución estándar
                                   de la variable dependiente para cada uno de los niveles de las covariables. 
                                   El tamaño de la muestra es de 14.632 observaciones.
                                   \end{tablenotes}
                                   \end{minipage}
