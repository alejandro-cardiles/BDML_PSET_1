\centering \footnotesize \setlength{\tabcolsep}{4pt} \renewcommand{\arraystretch}{0.8}
\begingroup
\fontsize{7.0pt}{8.5pt}\selectfont
\begin{longtable}{lrrrrrrr}
\toprule
  & Promedio & SD & Min & P10 & Mediana & P90 & Max \\ 
\midrule\addlinespace[2.5pt]
{\bfseries Ingresos laborales por hora} & 8.578.87 & 13.902.31 & 0.47 & 2.333.33 & 4.855.29 & 17.012.92 & 350.583.3 \\ 
{\bfseries Edad} & 39.08 & 13.11 & 19 & 23 & 37 & 58 & 91 \\ 
{\bfseries N. menores de edad en el hogar} & 0.84 & 0.97 & 0 & 0 & 1 & 2 & 7 \\ 
{\bfseries N. adultos mayores inactivos en el hogar} & 0.13 & 0.39 & 0 & 0 & 0 & 1 & 4 \\ 
\bottomrule
\end{longtable}
\endgroup

NA
NA
\begin{minipage}{0.8\textwidth}
                                   \begin{tablenotes}[scriptsize,flushleft]
                                   \item \footnotesize \textit{Nota:} Esta tabla muestra las estadísticas descriptivas de las variables continuas incluidas en el conjunto de datos.
                                   Las columnas de Promedio y SD muestran la desviación estándar de las variables. Mientras que las columnas P10, Mediana y P90 muestran los valores de la distribución en los percentiles 10, 50 y 90.
                                   Las columnas Min y Max muestran el minímo y el máximo valor muestral.
                                   El tamaño de la muestra es de 14.632 observaciones.
                                   \end{tablenotes}
                                   \end{minipage}
